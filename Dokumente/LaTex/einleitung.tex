\section{Einleitung}
Weltweit werden in fast allen Programmen, die wir alltäglich nutzen, Datenbanken verwendet. Überall treffen wir auf sie – sei es innerhalb unserer Freundesliste, die unsere Kontakte abspeichert,  auf Websiten, die unsere Daten verwalten oder bei Streamingportalen, die unsere Verläufe sichern. 

Es gibt diverse Formen von Datenbanken. Neben der hierarchischen oder objektorientierten wird innerhalb dieser Bachelorarbeit insbesondere eine relationale Datenbank genutzt, in die ein Knowlegde Graph umgewandelt und anschließend mit einem weiteren Graphen erweitert wird. Dies soll dazu dienen, zwei Datenquellen miteinander zu vereinen, um die Möglichkeit zu schaffen, sich aus mehreren  Quellen zu bedienen. 
Dafür werden die Datenbanken  wie das Digital Biblography \& Library Project, kurz DBLP, der Universität Trier und der Microsoft Academic Knowledge Graph verwendet. Beide Graphen beinhalten Publikationen, Facharbeiten, Konferenzen und Fachbücher, von denen letzteres Zitationen umfasst. 
Die Datenbank der Universität Trier umschließt nur die Daten im Bereich Informatik, welches diese um einiges verringert. Innerhalb dessen werden 5,2 Millionen Puplikationen gespeichert. Im Gegensatz dazu umfasst der Microsoft Graph keinen Fachbereich und hat mit 209,7 Millionen eine größere Auswahl an Puplikationen als die DBLP. (Fußnote von wann die Zahlen sind)

Zu Beginn dieser Arbeit wird zunächst die verwendeten Knowledge Graphen erklärt, die für das gesamte Vorhaben benötigt werden. Anschließend wird einer dieser Graphen, die DBLP, extrahiert und in einer eigens angelegten relationalen Datenbank gespeichert. Dazu wird ein eigener Parser entwickelt und eine Datenbank konstruiert, um diese universell - für Daten im Bereich von Publikation und Zitation – zu nutzen. 
Nach der Extraktion werden schließlich die extrahierten Daten vom Microsoft Academic Knowledge Graph erweitert. 
Das darauffolgende Ergebnis dieser Anwendung soll eine Datenbank zeigen, die alle Publikationen im Bereich Informatik  und die dazugehörigen Zitate enthält. 

Das Resultat dieser Arbeit wird im Fazit reflektiert und könnte innerhalb des Ausblicks auf zukünftige Anwendungen optimiert werden. 

\newpage
