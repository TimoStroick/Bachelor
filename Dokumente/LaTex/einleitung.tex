\section{Einleitung}
Weltweit werden in fast allen Programmen die wir alltäglich benutzen Datenbanken verwendet, sei es unsere Freundesliste, eine Website und sogar beim Einkaufen, überall treffen wir auf sie. Es gibt viele Formen von Datenbanken wie zum Beispiel relationale, hierarchische, objektorientierte Datenbanken und noch viele mehr. In dieser Bachelorarbeit wird ein Knowledge Graph in eine relationale Datenbank umgewandelt und anschließend noch mit einem weiteren Graphen erweitert. Dabei stellt sich natürlich die Frage was ist ein Knowledge Graph überhaupt. Grundsätzlich ist es auch eine Form von Datenbank. Man kann sich die Datenbank am besten so vorstellen das man Daten so speichert das man sie leichter suchen kann. Deshalb werden Daten meist mit Themen Gebieten zusammen verbunden. Wie bei Suchmaschinen wird mit einem Schlüsselwort gesucht und in dieser Art von Datenbank stehen alle Daten die etwas damit zu tun haben in direkter Verbindung mit dem Schlüsselwort. So lässt es sich einfacher und schneller in der Datenbank suchen. Im Gegensatz dazu steht eine relationale Datenbank in die der Knowledge Graph eingefügt wird. Hierbei steht nicht das suchen im Vordergrund, sondern in welcher Beziehung die Daten zueinander stehen. Damit ist der Grundaufbau sehr Unterschiedlich und das ist die Aufgabe die es gilt zu überwinden. Die Knowledge Graphen die benutzt werden sind der Microsoft Academic Knowledge Graph und das Digital Bibliography \& Library Project (DBLP) der Universität Trier. Beide Graphen beinhalten Publikationen, Facharbeiten, Konferenzen und Fachbücher. Die Datenbank der Universität Trier beinhaltet nur Daten im Bereich der Informatik was die Daten Anzahl um einiges verringert. Hier sind nur grob 5,2 Millionen Publikationen vertreten. Im Gegensatz da zu hat der Microsoft Academic Knowledge Graph keinen Fachbereich und ist international Vertreten. Daraus ergeben sich dann die 209,7 Millionen Publikationen in der Datenbank und das alleine im Jahr 2018. Für den ersten Teil werden die DBLP Daten in eine relationale Datenbank eingefügt. Dafür wird eine Datenbank entwickelt und anschließend alle Daten einzeln extrahiert und gespeichert. Dies machen wir mit der DBLP da es zunächst weniger Daten sind und in der DBLP keine richtigen Zitate vorhanden sind. Der zweite Teil ist nun das erweitern. Hier wird der Microsoft Academic Knowledge Graph verwendet da er zu den ganzen Publikationen auch noch 146 Millionen Zitation enthält. Für diesen Zweck erhält die Datenbank zusätzlich eine Zitat Beziehung wo die Daten eingefügt werden können.
